\documentclass{standalone}
\usepackage{flecken}
\begin{document}

\newcounter{counter}
\newcounter{x}
\setcounter{x}{7}
\begin{tabularx}{\textwidth}{*{15}{C}}
    \forloop{counter}{0}{\value{counter} < \value{x}}{&} \textcolor{black}{$\downarrow$} \\
    \forloop{counter}{0}{\value{counter} < 14}{\centering\number\value{counter} &}14
    \\ \hline
    \ccol{1} &
    \ccol{3} &
    \ccol{9} &
    \ccol{13} &
    \ccol{16} &
    \ccol{18} &
    \ccol{22} &
    \ccol{27} &
    \ccol{31} &
    \ccol{34} &
    \ccol{41} &
    \ccol{53} &
    \ccol{62} &
    \ccol{67} &
    \ccol{74}
    \\ \hline
    \forloop{counter}{1}{\value{counter} < \value{x}}{&}
    &
    \multicolumn{2}{l}{\textcolor{black}{$= x$}}
\end{tabularx}

\end{document}